% vim: spell spelllang=ca:
\documentclass[12pt, oneside]{article}
\usepackage[a4paper, left=2.5cm, right=2.5cm, top=2.5cm, bottom=2.5cm]{geometry}

\usepackage[T1]{fontenc}
\usepackage[catalan]{babel}
\usepackage{fontspec}
\setmonofont[Scale=MatchLowercase]{DejaVu Sans Mono}

\usepackage[colorlinks]{hyperref}

\usepackage{xcolor}
\usepackage{minted}
\definecolor{codeBg}{HTML}{FFFDED}
\setminted[shell-session]{
    % style=monokai,
    frame=single,
    framesep=3mm,
    % linenos,
    breaklines=true,
    encoding=utf8,
    fontsize=\footnotesize,
    bgcolor=codeBg
}

\title{
    \bfseries
    Administració de Sistemes Operatius \\
    \Large
Segona pràctica \\
\large
Insta\lgem ació d'aplicacions
}
\author{
Aleix Boné Ribó
}

\newcommand{\preglim}{6}
\newcommand{\preguntaminted}[1]{
    \ifnum#1<\preglim
        \inputminted[firstline=#1, lastline=#1]{shell-session}{commandes}
        \vspace{-2em}
    \else
        \vspace{1em}
    \fi
}

\newcommand{\pregunta}[1]{
    \vspace{2em}
\item #1 \preguntaminted{\theenumi}
    \addcontentsline{toc}{section}{Pregunta \theenumi}
}

\newcommand{\preg}[2][0]{
    \ifnum#1=0
        \inputminted[firstline=#2, lastline=#2]{shell-session}{preg6}
    \else
        \inputminted[firstline=#1, lastline=#2]{shell-session}{preg6}
    \fi
    \vspace{-2em}
}

\begin{document}

\maketitle

\hrulefill

\vspace{2em}

Actualment tenim un servidor d'aplicacions amb la versió 11 de Java, ja que és
necessària per poder executar l'aplicació de facturació que es fa servir a
l'empresa. Un dels usuaris en diu que necessita insta\lgem ar una altra aplicació
de gestió de clients que requereix la versió 14. Indica com gestionaries aquesta
situació sabent el següent:

\begin{enumerate}
    \item Volem que el canvi sigui el menys disruptiu possible

    \item És imprescindible que tots els usuaris continuïn utilitzant el software actual sense problemes

    \item Volem donar un mecanisme a l'usuari que demana aquesta nova aplicació que pugui utilitzar les dues aplicacions a la vegada

    \item Indica com migraries gradualment l'ús de l'aplicació vella per la nova
\end{enumerate}

\hrulefill

\pagebreak

\subsection*{Insta\lgem ació}

Assumint que estem utilitzant \emph{debian buster} i tenim insta\lgem at java 11 a través
dels binaris precompilats de la distribució (\texttt{openjdk-11-jre} insta\lgem
at a \texttt{/usr/lib/jvm/java-8-openjdk}).

Per insta\lgem ar la nova versió tenim dues possibilitats: insta\lgem ar a partir de
binaris precompilats o a partir de binaris autoinstalables. La versió 14 de java
no es troba al repositori de paquets de \emph{debian buster} i hauríem d'afegir un nou
repositori si volem fer el primer mètode, com que volem que el canvi sigui el
menys disruptiu seguirem el segon mètode.

Així doncs, insta\lgem arem la nova versió de java al directori
\texttt{/opt/java14}. Podem descarregar la versió 14 de java de la pàgina web de
OpenJDK: \url{https://jdk.java.net/archive/} (o la versió comercial d'oracle).
La instalacio simplement consisteix en descomprimit els continguts a la carpeta
de destí (en aquest cas \texttt{/opt}):

\inputminted{shell-session}{download}

Els binaris de java 14 es troben a \texttt{/opt/jdk14.0.2/bin}. L'usuari que vol
utilitzar la nova versió pot afegir aquest directori al principi del seu
\texttt{PATH} i utilitzar java 14 en comptes de 11. Si volgués utilitzar 11
hauria de cridar la binaria de java amb el path complet: \texttt{/usr/bin/java}

Per facilitar la crida, podem afegir enllaços simbòlics a java14 a
\texttt{/usr/bin}:

\inputminted{shell-session}{links}

D'aquesta manera, no cal que l'usuari modifiqui el \texttt{PATH} i pot cridar
els binaris de java 14 com a \texttt{java-14}, \texttt{jar-14}, \dots.

\pagebreak
\subsection*{Migració}

Per migrar la resta d'usuaris del sistema a java-14, podem moure l'actual
enllaç de \texttt{/usr/bin/java} a \texttt{/usr/bin/java-11} i crear un nou
enllaç \texttt{/usr/bin/java} a un que apunti a \texttt{/opt/jdk14.0.2/bin/java}.
Això faria que per defecte el sistema utilitzés java 14 i la versió antiga
encara es pogués accedir. El canvi es fàcil de revertir si es detectés alguna
incompatibilitat amb la nova versió.

Tot i que el mètode descrit funcionaria, \emph{debian} ens proporciona un sistema per
administrar la creació de symlinks a diferents versions d'un mateix programa
documentat a \texttt{update-alternatives(1)}. El sistema consisteix en crear un
enllaç a \texttt{/usr/bin/java} apuntant a \texttt{/etc/alternatives/java} que
al seu torn es un enllaç a la binaria de java que volem. Això permet tenir tots
els programes amb alternatives a la carpeta \texttt{/etc/alternatives} i
gestionar-los mes fàcilment. En el nostre cas, caldria afegir la nostra
insta\lgem ació de java a les \texttt{update-alternatives}

\inputminted{shell-session}{aptinstall}

(S'hauria de realitzar el mateix procediment per la resta de comandes de java
necessàries: \texttt{jar}, \texttt{keytool}, \dots)

\end{document}
