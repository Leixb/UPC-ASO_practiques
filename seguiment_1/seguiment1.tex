% vim: spell spelllang=ca:
\documentclass[12pt, oneside]{article}
\usepackage[a4paper, left=2.5cm, right=2.5cm, top=2.5cm, bottom=2.5cm]{geometry}

\usepackage[T1]{fontenc}
\usepackage[catalan]{babel}
\usepackage{fontspec}
\setmonofont[Scale=MatchLowercase]{DejaVu Sans Mono}

\usepackage{hyperref}

\usepackage{xcolor}
\usepackage{minted}
\definecolor{codeBg}{HTML}{FFFDED}
\setminted[shell-session]{
    % style=monokai,
    frame=single,
    framesep=3mm,
    % linenos,
    breaklines=true,
    encoding=utf8,
    fontsize=\footnotesize,
    bgcolor=codeBg
}

\title{
    \bfseries
    Administració de Sistemes Operatius \\
    \Large
Primera pràctica \\
\large
Permisos i proteccions
}
\author{
Aleix Boné Ribó
}

\newcommand{\preglim}{6}
\newcommand{\preguntaminted}[1]{
    \ifnum#1<\preglim
        \inputminted[firstline=#1, lastline=#1]{shell-session}{commandes}
        \vspace{-2em}
    \else
        \vspace{1em}
    \fi
}

\newcommand{\pregunta}[1]{
    \vspace{2em}
\item #1 \preguntaminted{\theenumi}
    \addcontentsline{toc}{section}{Pregunta \theenumi}
}

\newcommand{\preg}[2][0]{
    \ifnum#1=0
        \inputminted[firstline=#2, lastline=#2]{shell-session}{preg6}
    \else
        \inputminted[firstline=#1, lastline=#2]{shell-session}{preg6}
    \fi
    \vspace{-2em}
}

\begin{document}

\maketitle

\addcontentsline{toc}{section}{Enunciat}
\begin{center}
Donat el següent estat d'un directori, respon a les preguntes que es plantegen
\end{center}

\inputminted{shell-session}{enunciat}

\begin{itemize}
    \item \textbf{Nota:} Per defecte un usuari només pertany al grup que té el mateix nom que ell.

    \item \textbf{Nota 2:} Assumeix que cada pregunta és independent, o sigui que l'efecte de les comandes NO es propaga a les altres preguntes.
\end{itemize}

\pagebreak

\begin{center}
    Respon JUSTIFICANT cada resposta a les següents preguntes
\end{center}

\begin{enumerate}

    \pregunta{Indica si funcionaria la següent comanda:}

    No funcionaria completament:

    L'usuari \texttt{profe} té permisos de lectura, escriptura i execució
    (accés) al directori \texttt{d1}. Per tant pot llistar els continguts del
    directori i eliminar tots els fitxers. Al tenir la \texttt{flag} \texttt{-f,
    --force} no demana confirmació al eliminar els fitxers als que no té permís
    d'escriptura (\texttt{f1} i \texttt{f2}).

    El que no pot fer és eliminar el directori \texttt{d1}, ja que al formar
    part del grup \texttt{profe}, no té permisos d'escriptura a la carpeta
    \texttt{/shared}.

    Així doncs, la comanda eliminaria els fitxers continguts a \texttt{d1} i
    fallaria al eliminar el directori \texttt{d1}. Quedaria el directori buit,
    però sense eliminar.

    \pregunta{Quin és l'efecte d'executar:}

    Com que l'arxiu ja existeix en el sistema, no s'ha d'afegir com a entrada al
    directori \texttt{d1}, per tant només necessitem permisos d'escriptura en el
    fitxer \texttt{d1/f1} i d'accés al directori (només cal permís d'execució,
    el permís de lectura no és necessari ja que no cal llistar els continguts de
    la carpeta).

    En aquest cas l'usuari \texttt{student} té els permisos adients i per tant
    la comanda s'executarà amb èxit afegint una línia amb \texttt{Testing} al
    final de l'arxiu \texttt{d1/f1}.

    \pregunta{Tindria èxit executar:}

    Sí. L'usuari \texttt{profe} té permisos d'escriptura al directori
    \texttt{d1}, per tant pot realitzar la comanda. Tot i així, al no tenir
    permisos d'escriptura al fitxer \texttt{d1/f2}, la comanda preguntarà si vol
    sobreescriure el fitxer al usuari (ja que no s'ha especificat la \emph{flag}
    \texttt{-f, --force}).

    \pagebreak
    \null
    \vspace{1em}

    \pregunta{Quins permisos, propietari i grup tindria un fitxer creat amb
    aquesta comanda?}

    Tindrà permisos 644 (\texttt{rw-r--r--}) i usuari i grup \texttt{profe}.
    Per defecte la comanda \texttt{touch} crea fitxers amb permisos 666. Al
    aplicar la \texttt{umask} de 022\footnotemark, els permisos resultants són 644.
    L'usuari i grup són ambdós \texttt{profe} ja que la comanda s'ha executat
    amb aquest usuari.

    \footnotetext{Assumint que la \texttt{umask} de l'usuari \texttt{profe} és
    igual a la de l'usuari \texttt{aso} que es mostra a l'enunciat.}

    \pregunta{Funcionaria la següent comanda?}

    No. El directori \texttt{d2} té el bit \emph{sticky} activat (\texttt{t}),
    per tant, només els propietaris dels arxius (i \texttt{root}) poden
    eliminar-los del directori. El propietari de \texttt{d2/file} no és
    \texttt{rserral}, per tant la comanda mostra un error indicant que no es pot
    eliminar el fitxer \texttt{d2/file} i acaba. No es produeix cap modificació
    al sistema de fitxers ja que no hi ha cap fitxer pertanyent a
    \texttt{rserral} dins el directori i aquest no es pot eliminar ja que no
    està buit.

    \vspace{2em}

    \pagebreak

    \pregunta{Volem oferir un directori del nostre home:
        \texttt{/home/aso/shared\_folder} a tots els usuaris del grup
        \texttt{users}, però al mateix temps volem restringir que puguin accedir
        de forma senzilla al directori \texttt{/home/aso} i als altres
        subdirectoris que hi pugui haver. Indica la millor manera de poder
        aconseguir això modificant els permisos dels directoris que creguis
    oportú.}

    \vspace{1em}

    Modificarem els permisos de \texttt{/home/aso} per eliminar el permís de
    lectura, deixant-lo en: 751 i amb propietari i grup \texttt{aso}. D'aquesta
    manera els usuaris del grup \texttt{users} només poden accedir al directori
    però no llistar el seu contingut.

    \preg[1]{2}

    Al no poder llistar el contingut només poden accedir a les carpetes i arxius
    de les quals tinguin coneixement. Aquest tipus de seguretat per ofuscació
    no és gens segura, ja que és molt probable que tinguem algunes carpetes com
    \texttt{Documents} o fitxers \texttt{\textasciitilde/.bashrc} fàcils
    d'endevinar. Per assegurar-nos que no es pugui accedir a cap carpeta o
    fitxer hem de canviar els permisos de tots els fitxers de
    \texttt{/home/aso}.

    \preg{3}

    Això soluciona el problema per els fitxers actuals, si volem evitar que futurs
    nous fitxers creats pugin ser llegits haurem d'utilitzar una \texttt{umask}
    que restringeixi tots els permisos als usuaris que no pertanyin al grup del
    fitxer (\texttt{o-rwx} / \texttt{o=}).  Per que aquest canvi sigui persistent l'hem
    d'afegir al \texttt{\textasciitilde/.bashrc} o equivalent.

    \preg{4}

    Un cop restringit l'accés a totes les carpetes, modifiquem la carpeta
    \texttt{shared\_folder} canviant el seu grup a \texttt{users} i donant
    permisos \texttt{rwx} al grup a la carpeta:

    \preg[5]{6}

    Al ser una carpeta compartida, és possible que ens interessi activar el bit
    \emph{sticky} per evitar que els usuaris puguin eliminar arxius de la resta.

    \preg{7}

\end{enumerate}

\end{document}
